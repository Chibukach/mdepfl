\documentclass[11pt]{article}
%EPFL Logo and Mention in completion of semester project
\title{\textbf{Music Information Retrieval based on the Free Music Archive Dataset}}
\author{Chibueze Ukachi\\
		Micha\"el Defferrard\\
		Pierre Vandergheynst}

\date{09/06/2017}
 
\begin{document}

\maketitle
\newpage
\section*{Acknowledgements}
Thanks to Michael for patience and support and thanks to Prof Pierre for access to LTS2. Also, thanks to FMA and artists.

\newpage

\section*{Abstract}

Brief description of the FMA dataset and why we chose it. My proposed methodology and features that I used. Max accuracy achieved. 

\newpage

\section*{List of Tables}
Table figure numbers and title

\section*{Figures}
The title of any figures and its corresponding figure number. 
\newpage

\tableofcontents
\newpage

\section{Introduction}
Overview of report including aims and objectives
\newpage

\section{Free Music Archive Dataset}
what the data set is, breakdown of it and how it is structured
\newpage
\section{Music Information Retrieval and Genre Recognition}
what MIR is and why it is important today
explain what Genre recognition is and the main challeges associ
Techniques tried in the past and their drawbacks 
Modern techniques
State of the art techniques today. 
\newpage

\section{Methodology}
\subsection{Song Retrieval}
explain processing of extracting song from data set
\newpage

\subsection{Feature Selection}
explain mfcc, zcr, onsets and spectral centroid
\newpage

\subsection{Machine Learning}
brief overview of algorithms used
\newpage

\newpage
\section{Experiments}
The different approaches that I tried.
\newpage
\section{Results}
All tables, figures and explanations

\newpage

\section{Limitations and Future Improvements}
Drawbacks of the strategy and things that can be done better
\newpage

\section{Conclusion}
Summary and overview. Also, what I think was the most effective strategy
\newpage

\section{Bibliography}
Complete bibliography
\emph{ref1}\footnote{http://links to refenrences } 

\end{document}
